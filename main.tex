% Created 2023-08-21 Mon 10:59
% Intended LaTeX compiler: pdflatex
\documentclass[11pt,a4paper,final]{article}
\usepackage[a4paper, total={7in, 10in}]{geometry}
\usepackage{algorithm2e}
\usepackage{booktabs}
\usepackage{subcaption}
\usepackage{graphicx}
\usepackage{tikz}
\usepackage[utf8]{inputenc}
\usepackage[T1]{fontenc}
\usepackage{graphicx}
\usepackage{longtable}
\usepackage{wrapfig}
\usepackage{rotating}
\usepackage[normalem]{ulem}
\usepackage{amsmath}
\usepackage{amssymb}
\usepackage{capt-of}
\usepackage{hyperref}
\usepackage[T1]{fontenc}
\usepackage{lmodern}
\usepackage{standalone}                     % Allow standalone documents
\usepackage{subcaption}                     % Allow subfigures
\usepackage{subfloat}                       % Subfigures
\usepackage{lipsum}                         % Dummy filler text
\usepackage{amsfonts}                       % Cool math fonts
\usepackage{multicol}                       % Add capability to make columns
\usepackage{hyperref}                       % Cool clean hyperlinks
\setlength\parindent{0pt}                   % No indent for paragraphs
\usetikzlibrary{arrows.meta}                % Arrows for tikz
\renewcommand*{\sectionautorefname}{Section}
\renewcommand*{\subsectionautorefname}{Subsection}
\renewcommand*{\subsubsectionautorefname}{Subsubsection}
\renewcommand*{\paragraphautorefname}{Paragraph}
\renewcommand*{\algorithmautorefname}{Algorithm}
\newcommand{\Or}{\textbf{ or }}
\renewcommand*{\And}{\textbf{ and }}
\newcommand\mycommfont[1]{\footnotesize\ttfamily\textcolor{gray}{#1}}
\newcommand{\T}{\mathcal{T}}                % To make it clear the difference
\newcommand{\Tau}{T}                        % between Tau and T
\newcommand{\AC}{AC(u, d, v, \eta)}            % Set the parameters for AC once
\newcommand{\PC}{PC(u, d, v)}               % Set the parameters for PC once
\newcommand{\ACi}{AC(u_i, d_i, v_i, \eta_i)}% Set the parameters for AC once
\newcommand{\PCi}{PC(u_i, d_i, v_i)}        % Set the parameters for PC once
\newcommand{\Not}{\textbf{not }}            % Custom `not' operator
\newcommand{\visit}{(b_i, a_i, e_i, u_i, d_i, v_i, \eta_i, \xi_i)}
\newcommand{\I}{\mathbb{I}}                 % Set of visit tuples
\newcommand{\C}{\mathbb{C}}                 % Charger availability information
\newcommand{\U}{\mathcal{U}}                % Uniform distribution
\newcommand{\Sol}{\mathbb{S}}               % A shorthand for visit tuple
\newcommand{\M}{\mathbb{M}}                 % A shorthand for the metadata
\newcommand{\Hd}{\mathbb{H}}                % Set of discrete times
\newcommand{\Nu}{\mathcal{V}}               % Draw a nice Nu
\newcommand{\Iset}{\mathcal{I}}             % Set of visits 1-I
\newcommand{\Isetinit}{\mathcal{I}_0}       % Set of visits inital visits
\newcommand{\Isetfinal}{\mathcal{I}_f}      % Set of visits final visits
\newcommand{\Bset}{\mathcal{B}}             % Set of visits 1-B
\newcommand{\Qset}{\mathcal{Q}}             % Set of visits 1-Q
\newcommand{\Jset}{\mathcal{J}}             % Set of visits 1-J
\newcommand{\Jsetq}{\mathbb{J}}             % Set of visits 1-J for queue active times
\newcommand{\Hset}{\mathcal{H}}             % Set of visits 1-H
\newtheorem{definition}{Definition}[section]
\author{Alexander Brown}
\date{\today}
\title{Proposal For The Approach To The Simulated Annealing Fully Fuzzy Position Allocation Problem Utilizing Mixed Integer Linear Programming Constraints and Non-Linear Battery Dynamics}
\hypersetup{
 pdfauthor={Alexander Brown},
 pdftitle={Proposal For The Approach To The Simulated Annealing Fully Fuzzy Position Allocation Problem Utilizing Mixed Integer Linear Programming Constraints and Non-Linear Battery Dynamics},
 pdfkeywords={},
 pdfsubject={},
 pdfcreator={Emacs 29.1 (Org mode 9.6.7)}, 
 pdflang={English}}
\begin{document}

\maketitle
\tableofcontents

\parskip 3mm                                % Set the vetical space between paragraphs
\let\ref\autoref                            % Redifine `\ref` as `\autoref` because lazy
\SetCommentSty{mycommfont}                  % Set the comment color

\let\ref\autoref                            % Redifine `\ref` as `\autoref` because lazy

\section{Introduction}
\label{sec:introduction}
The public transportation system is crucial in any urban area; however, the increased awareness and concern of
environmental impacts of petroleum based public transportation has driven an effort to reduce the pollutant footprint
\cite{de-2014-simul-elect,xylia-2018-role-charg,guida-2017-zeeus-repor-europ,li-2016-batter-elect}. Particularly,
the electrification of public bus transportation via battery power, i.e., battery electric buses (BEBs), has received
significant attention \cite{li-2016-batter-elect}. Although the technology provides benefits beyond reduction in
emissions, such as lower driving costs, lower maintenance costs, and reduced vehicle noise, battery powered systems
introduce new challenges such as larger upfront costs, and potentially several hours long ``refueling'' periods
\cite{xylia-2018-role-charg,li-2016-batter-elect}. Furthermore, the problem is exacerbated by the constraints of the
transit schedule to which the fleet must adhere, the limited amount of chargers available, and the adverse affects in
the health of the battery due to fast charging \cite{lutsey-2019-updat-elect}.

Many recent efforts have been made to simultaneously solve the problems of route scheduling, and charging fleets and
determining the infrastructure upon which they rely, e.g., \cite{wei-2018-optim-spatio,sebastiani-2016-evaluat-elect,hoke-2014-accoun-lithium,wang-2017-elect-vehic}. Several simplifications are made to make these problems
computationally feasible. These simplifications to the charge scheduling model include utilizing only fast chargers
while planning \cite{wei-2018-optim-spatio,sebastiani-2016-evaluat-elect,wang-2017-optim-rechar,zhou-2020-bi-objec,yang-2018-charg-sched,wang-2017-elect-vehic,qin-2016-numer-analy,liu-2020-batter-elect}. If slow chargers are used,
they are only employed at the depot and not the station \cite{he-2020-optim-charg,tang-2019-robus-sched}. Some
approaches also simplify by assuming a full charge is always achieved
\cite{wei-2018-optim-spatio,wang-2017-elect-vehic,zhou-2020-bi-objec,wang-2017-optim-rechar}. Others have assumed
that the charge received is proportional to the time spent on the charger
\cite{liu-2020-batter-elect,yang-2018-charg-sched}, which can be a valid assumption when the battery state-of-charge
(SOC) is below 80\% charge \cite{liu-2020-batter-elect}.

The intent of the proposed work is to build upon the Position Allocation Problem \cite{qarebagh-2019-optim-sched}, a
modification of the well studied Berth Allocation Problem (BAP), as a means to schedule the charging of electric
vehicles \cite{buhrkal-2011-model-discr,frojan-2015-contin-berth,imai-2001-dynam-berth}. The BAP model is a rectangle
packing problem with the goal of allocating space for incoming vessels to be berthed and serviced as depicted in
\autoref{subfig:bapexample}. The BAP can be modeled both continuously and discretely both temporally and spatially
\cite{buhrkal-2011-model-discr,frojan-2015-contin-berth}. Furthermore, the demand to efficiently handle an
ever-growing demand for servicing cargo, with an estimated 1.24 billion of the 8.02 billion tonnes of all shipping cargo
in 2007 \cite{buhrkal-2011-model-discr}, provides a strong pool of robust literature to draw from. Methods of handling
multiple quays to handle general berthing scenarios \cite{frojan-2015-contin-berth,dai-2008-suppl-chain-analy},
heuristic procedures for quicker solve times have been introduced \cite{imai-2001-dynam-berth}, static (full time
horizon) and dynamic (rolling-time horizon) models have been created for daily and real-time solutions, respectively,
and even fuzzy set theory has been applied to allow for more flexible schedules \cite{bello-2019-fuzzy-activ}.

Furthermore, this proposed work intends to extend the work of the PAP's novel approach to BEB charger scheduling even
further. The PAP's reenvisioning of the BAP provides the stepping stones on which to further extend the utility of this
robust technology. The intended work to be introduced, much of which has been completed, promises much potential for
further research and development in regard to BEBs. What follows is a proposal for a Simulated Annealing (SA)
implementation of the PAP utilizing Fully Fuzzy Mixed Integer Linear Programming (FFMILP) constraints and nonlinear
battery dynamics.
\section{Proposed Project}
\label{sec:proposed-project}
\begin{subfigures}
    %%~~~~~~~~~~~~~~~~~~~~~~~~~~~~~~~~~~~~~~~~~~~~~~~~~~~~~~~~~~~~~~~~~~~~~~~~~~~~
    % BAP
    \begin{figure}[htpb]
    \centering
        \includestandalone{sup-doc/milp-pap-paper-frontiers/img/bap}
        \caption{Example of berth allocation. Vessels are docked in berth locations (horizontal) and are queued over
          time (vertical). The vertical arrow represents the movement direction of queued vessels and the horizontal
          arrow represents the direction of departure.}
        \label{subfig:bapexample}
    \end{figure}
    \hfill

    %%~~~~~~~~~~~~~~~~~~~~~~~~~~~~~~~~~~~~~~~~~~~~~~~~~~~~~~~~~~~~~~~~~~~~~~~~~~~~
    % PAP
    \begin{figure}[htpb]
    \centering
        \includestandalone{sup-doc/milp-pap-paper-frontiers/img/pap}
        \caption{Example of position allocation. Vehicles are placed in queues to be charged and move in the direction
          indicated by the arrow.}
        \label{subfig:papexample}
    \end{figure}
\end{subfigures}

\bibliographystyle{plain}
\bibliography{/home/alex/Documents/docs/proposal/citation-database/lit-ref,/home/alex/Documents/docs/proposal/citation-database/lib-ref,~/Documents/citation-database/lit-ref,~/Documents/citation-database/lib-ref}
\end{document}